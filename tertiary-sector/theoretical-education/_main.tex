\documentclass{Academic}
\usepackage{csquotes}

\addbibresource{references.bib}

\begin{document}
%Easy customisation of title page
%TC:ignore
    \myabstract{%\small
%\noindent\textbf{Abstract}
%%Abstract below
%
%The hypothesis/question and its context, scope and significance are briefly and precisely defined. The main conclusions are brief, precise and clear, noting methods used and key results. The language is clear, precise and easy to understand with no irrelevant information.}
    \renewcommand{\myTitle}{A Survey of Opportunities and Risks of AI Systems in Education}
    \renewcommand{\MyAuthor}{Leon Knorr}
    \renewcommand{\MyDepartment}{Mannheim Master of Datascience}
    \renewcommand{\ID}{1902854}
    \renewcommand{\Keywords}{Education, AI}
    \maketitle
%\vspace{-1.9em}\noindent\rule{\textwidth}{1pt} %add this line if not using abstract
    %\onehalfspacing
%TC:endignore


    \section{Introduction}
    Artificial Intelligence (AI) systems have made significant inroads into the education landscape, promising enhanced learning experiences, improved efficiency, and personalized educational pathways. As the deployment and research of AI technologies in education continues to gather momentum, it is essential to acknowledge that the impact of these systems extends far beyond their immediate advantages and drawbacks \cite{chen_ai_2023}. Engaging in a meaningful and informed discourse about AI in education is imperative to ensure the fair and responsible implementation and development of AI systems in an education setting. In addition, a survey of the Pew Research Center revealed, that a growing share of people are concerned about the development and deployment of AI systems. This includes the fear of potential job loss, privacy concerns and the loss of human connection \cite{nadeem_1_2022}. In this work, Opportunities and Risks are shown based on current state-of-the-art AI systems while addressing, widespread concerns, as well as showing potential problems which should be addressed in the current rapid development of such systems.
    
    \section{Approach}
    In this work, the Opportunities and Risks of AI systems in Education will be discussed in two focus topics:
    \begin{enumerate}
        \item Teaching
        \item Assistive Technologies
    \end{enumerate}
    First, the current state of adoption of AI systems in these categories are laid out. Then, the opportunities of such systems are presented based on real world examples and their future potential. Then, the risks of AI systems and current research in the context of education is discussed, and possible solutions are presented.
    
    \section{The current state of AI and Technology in Education}
    In this section, the current state of AI and Technology is presented. First, the current technologies used in Teaching are laid out. Then, the assistive technologies deployed at schools are presented.

    \subsection{Current Technologies in Teaching}
    Quantifying the actual usage of technological advances in teaching is hard, as it is very subjective to the teacher, as well as the teaching environment. Since 2013, the International Association for the Evaluation of Educational Achievement (IEA) conducts the International Computer and Information Literacy Study in a cycle of 5 years. This study has been designed to capture the current state of how well students and teachers a like are prepared for study, work and life in a increasingly digital world \cite{noauthor_icils_nodate}. Its latest evaluation cycle has been conducted in 2018 with the next one starting in 2023. The study, as well as a German counterpart, showed, that the availability of Technology and digital devices vary greatly between schools, students and teachers \cite{musmann_digitalisierung_2021}. In 2018, most teachers and students didn't have their own portable device for school usage. These circumstances got better over the course of the pandemic. However, still around 60\% of teachers claim to not have access to their own portable digital device for teaching \cite{musmann_digitalisierung_2021}, with 30\% of schools not providing Wireless internet access to teachers or students. This goes inline with the usage numbers for Information and Communication Technology (ICT) during lessons, only 43\% of teachers use word processing or presentation software programs, with 32\% using digital content to supplement school materials \cite{noauthor_icils_nodate}. Besides availability, most teachers claim, that missing confidence in the usage of digital devices and tools is one of the main reasons for the missing integration of ICTs. This applies especially for the usage of learning management systems or Education cloud software such as Moodle or ILIAS \cite{noauthor_icils_nodate}. As a result, the usage of such systems remains very sparse for in presence lectures. An exception is given, for gamification software, e.g. Quizlet or Kahoot, as 60\% of schools report frequent use and good availability \cite{noauthor_icils_nodate}. Because of these circumstances, it comes at no surprise, that the use of photocopiers, paper and overhead projectors are still the most used technologies in teaching \cite{noauthor_202122_nodate}.

    \subsection{Assistive Technologies in the Classroom}
    The use of assistive technologies in the classroom is key to achieve an inclusive and fair environment, where students with learning disabilities have the same opportunities and chances as students without disabilities. Learning disabilities can range from visual impairment, such as red-green weakening or colorblindness, over learning disorders, e.g.: Dyslexia (reading disability) or Dysgraphia (writing disability) to listening related issues \cite{adebisi_using_2015}. With that, the field of possible learning disabilities is quite vast and can have a large effect on a students' ability to follow a lesson. Assistive technologies can mitigate the effects of a students disabilities, by enhancing students basic skills, however, they can't replace them and thus can only provide maximal assistance if they are integrated into the educational process \cite{adebisi_using_2015}. At the same time, for students with disabilities, they are an essential work tool as a pen might be for others and are thus needed, so that a disabled student can participate equally with their learning peers in an educational environment. \\
    As every student has individual needs, the use of assistive technologies in education needs a lot of collaboration, from both teachers, parents, students and medical personnel, to determine the right tools for a students specific problems \cite{adebisi_using_2015}. Despite diagnosis and funding, just providing assistive technologies isn't enough, teachers also need to be able to provide assistance, teach the child to use assistive technologies before they are required and need to be technologically trained. Technological training is very important, to provide technological support if needed and to be able to incorporate the students needs in the learning plan \cite{adebisi_using_2015}. In addition, it is important, that learning plans take assistive technologies and learning disabilities into account from the beginning, so that students with disabilities don't fall behind even before their diagnosis. \\
    As technological training has been determined to be one of the most important factors and the availability of digital devices in schools are sparse, the usage of assistive technologies in traditional classrooms is quite rare. Teachers aren't sufficiently technologically trained to use assistive technologies on a broad scale, as most don't feel comfortable using technology in general regularly in lessons \cite{noauthor_icils_nodate}. Instead of using assistive technologies, teachers rely on practices such as \cite{teaching_learning_dissabilities}:
    \begin{itemize}
        \item the assistance in identifying potential tutors or note-takers
        \item allow for the extensions on assignments and essays
        \item allow preferential seating
        \item arranging additional meetings to discuss specific learning needs
        \item allow extra time on exams
        \item provide separate distraction free environments during exams
        \item allow for the usage of screen enhancement software if computers are needed during lessons
    \end{itemize}
    What all of these have in common is, that they are all exclusive and separate a student with special treatment from its learning peers. And as photocopiers and overhead projectors are still the most used technologies during classes, the use of assistive technologies is rarely applicable, without special treatment, such as giving only this student access to a digital device with optical character recognition, as the learning material isn't available in digital form.

    \subsection{Defining AI in Education}
    Defining AI itself is complicated. As of today, there is no precise and universally accepted definition of AI available \cite{Stanford_AI}. One way to define AI is as \enquote{The activity devoted to making machines intelligent, and intelligence is that quality that enables an entitiy to function appropriately and with foresight in its environment.} \cite{Stanford_AI}. This definition depends on whether a system can be credited to function appropriately and with foresight, which can be very subjective. This can range from accepting simple \enquote{dumb} programs such as calculators as AI. In recent times, the understanding of AI has shifted, as machine and deep learning techniques become increasingly available to the public. Today, when talking about an AI, most people think of Computer Vision systems, Large Language Models or intelligent speech recognition agents such as Amazons Alexa or take influence from the entertainment sector \cite{nader_public_2022}.

    The definition of AI in education builds upon this general definition, by defining AI systems in Education (AIED) as systems, that include \enquote{intelligent education, innovative virtual learning, data analysis and prediction} \cite{chen_artificial_2020}. Intelligent education systems are designed to improve learning value and efficiency by providing timely and personalized instruction and feedback to both tutors and students. They include techniques for learning analysis, recommendation and knowledge understanding and acquirement, which uses sets out to use current state-of-the-art data mining, analytics and machine and deep learning technologies \cite{chen_artificial_2020}.

    \section{Opportunities \& Risks}
    Artificial intelligence in education is said to have a lot of possibilities. While some suggestions are coming right out of science-fiction, others are already in use or being tested. In this section, the opportunities and risks of AI stated by different publications are presented and discussed. First, the focus will be on Artificial Intelligence in teaching, followed by a discussion about how AI could transform assistive technologies.

    \subsection{Teaching}
    In teaching, AI opens up a plethora of opportunities, from personalized learning experiences, to data and analytics for teachers. AI has the potential to transform the world of teaching, while shining a spotlight onto issues that have been overlooked or ignored for years \cite{holmes_artificial_2023}, giving the opportunities to improve on those problems.

    \subsubsection{Collaborative Learning}
    In Collaborative learning, students work together in groups to solve a set of problems, which has been proven to be an effective strategy to positively influence learning results \cite{holmes_artificial_2023}. One of the most important processes in Collaborative learning is the group formation process. According to the Belbin theory, all students show behavior, which can be mapped to one of 8 team roles, which facilitate the progress of the whole team. The behavioral pattern of each team role is by six factors: personality, mental ability, current values and motivation, field constraints, experience, and role learning \cite{alberola_artificial_2016}. It has been shown, that it is essential for a group to perform well, to be composed of a balanced combination of the team roles. In order to achieve this, students were asked to take self-tests, resulting in the assignment of a team role and students were then matched based on these results, which results in a high organizational effort and can be biased, as individuals may have a preconceived image of themselves \cite{alberola_artificial_2016}. Artificial Intelligence can be used to solve these problems as shown by Alberola et al. \cite{alberola_artificial_2016}, who proposed a team formation mechanism based on student feedback, coalitional structure generation, and Bayesian learning. Instead of self-tests, the mechanism uses an iteratively updated knowledge base consisting of student feedback about their peers most relevant team role. As a result, first the mechanism assigns random groups, and after feedback has been given, uses this information to assign each student $a_i$ to their most probable team role and composes a balanced group. In order to extract information from student feedback, Alberola et al. \cite{alberola_artificial_2016} applied Bayesian learning in order to estimate the probability of a student having team role $r'$ as its most relevant role given the history $H$ of student evaluations, such that:
    \begin{equation}
        p(role_i = r'_i|H) = \frac{p(H|role_i = r'_i)p(role_i=r'_i)}{\sum_{r \in R} p(H|role_i = r)p(role_i=r)}
    \end{equation}
    Last but not least, to solve the problem of team formation, they showed, that student team formation can be expressed as a coalition structure generation problem, which aims at partitioning the components of a set into exhaustive and disjoint coalitions so that the global benefits of the system are optimized \cite{alberola_artificial_2016}. This problem can then be solved using linear programming. Such systems, can ensure that each student will be assigned to the most effective group of students possible, increasing the chance of achieving satisfactory results. However, such systems are not able to capture social relationships and only focus on student performance, rated by their peers. As a result, matching students who hate each other might create tension inside the group and hinder progress. In addition, students might also not always act faithful, and fill out the evaluation for students they like and hate differently. However, these circumstances could present students with opportunities to form new social relationships or solve social problems.

    \subsubsection{Student Forum Monitoring}
    During the pandemic and after, the usage of learning management systems has increased. One feature of these systems, is a \enquote{student forum}, where students can create posts to ask questions. Other students, as well as tutors, can then traverse through these posts to find information or to answer questions posted by others. In a school classroom setting, the student size is quite small and manageable. However, for higher education institutions and distance learning universities, the number of students per course can grow into multiple hundreds \cite{goel_using_2017}. Monitoring and managing forums with such a high number of participants can be challenging, and can lead to unanswered questions and bad student morale. AI poses the opportunities to alleviate this problem, by employing technologies like Large Language Models (LLMs) or intelligent agents, to act as a filter or preprocessor of forum posts \cite{goel_using_2017,holmes_artificial_2023}. In this setting, AI systems answer simple posts automatically, such as questions like: \enquote{When is the submission X due?}, and classify each posts into relevance categories and group them together. A tutor is then only notified about a group of posts, which is only needed to be answered once, or if posts need special attention. This process can also be supported by sentiment analysis models, which monitor each post, in order to reveal negative or non-productive student emotional states \cite{holmes_artificial_2023}. As of now, learning management systems do not employ any of the mentioned techniques, however, recent research has proven LLMs to be effective for content moderation \cite{wang_evaluating_2023}. LLMs have been fine-tuned or prompted to detect online hate speech, which is content that expresses hate or encourages violence towards a person or group based on race, religion, gender, or other identity characteristics \cite{wang_evaluating_2023}. An important factor for such systems, is the ability to generate fitting and meaningful explanations, as to why a specific piece has been flagged. However, many systems lack this ability, except for the latest GPT models \cite{wang_evaluating_2023}. In addition, many LLMs and models suffer from biases and might thus provide misleading and false accusations to moderators. As a result, even though in theory, the application of forum monitoring and moderation systems is argued to have a high potential for both students and tutors, as information is available more quickly, the overall workload is reduced and unconfident teachers are supported, but the practical limitations of current technology make the deployment and integration of such systems difficult.

    \subsubsection{AI Learning Companions}

    \subsubsection{Continuous Assessment}
    Current educational systems around the world rely on \enquote{high-stake} end-of-course examinations. Despite, that there is little evidence for their reliability, validity or accuracy and the fact that psychologists proved that it is wrong to make decisions based upon a single test score, there has been no change in the way students are examined for centuries \cite{holmes_artificial_2023}. This behavior, resulted in educators all too often teaching content for the test, instead of for the learning experience and the greater outcome of the course. AI driven learning management systems have the opportunity to change this, by providing a continuous Assessment. If AI becomes an always present tool, used by students throughout the process, being as a learning companion, an automatic feedback system or as the backbone of a learning management tool. The captured information about the student throughout its learning process could be used to create a detailed intelligent personal resumé of the students learning experiences, which can be used to assess whether a student has achieved mastery in the course or not \cite{holmes_artificial_2023}. Because this assessment happens in the background, at all times, it is almost impossible to cheat or subvert the system's intention. This would also alleviate the impact of \enquote{bad days} or disadvantageous personal situations, as it acts as a moving average over the course of the learning process. It also allows for the early identification of problems by teachers, which can be addressed directly, as they are reflected in the resumé \cite{holmes_artificial_2023}. As of the time of writing, no research in this area could be found, matching with Holmes et al.'s claims, that current research only focuses on \enquote{How to make the current examination process safer} \cite{holmes_artificial_2023}.\\
    Despite the potential advantages of such a system, it raises serious ethical concerns about data usage, privacy, security, reliability and fairness, which all need to be thoroughly considered and discussed. For example:
    \begin{itemize}
        \item How is data used?
        \item Where is it stored?
        \item How is it stored?
        \item Who has access to what kinds of data?
        \item What kind of data is considered?
        \item What biases are introduced in the model?
        \item How does the model come to a given conclusion?
        \item How is a given result explained?
    \end{itemize}
    These are just a few examples of the topics which have to be considered, before and during the development of such a continuous assessment system. Again, AI systems have a great potential to transform education in the regard of how students are assessed, but they also leave a great amount of questions unanswered, which are essential to ensuring a fair and acceptable result.

    \subsubsection{Teaching Assistants}

    \subsection{Assistive Technologies}

    \section{Conclusion}

%TC:ignore
%\clearpage %add new page for references
    \singlespacing
    \emergencystretch 3em
    \hfuzz 1px
    \printbibliography[heading=bibnumbered]

% \clearpage
% \begin{appendices}

% \section{Here go any appendices!}

% \end{appendices}

%TC:endignore
\end{document}