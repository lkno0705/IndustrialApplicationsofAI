\documentclass{Academic}
\usepackage{csquotes}

\addbibresource{references.bib}

\begin{document}
%Easy customisation of title page
%TC:ignore
    \myabstract{\small
\noindent\textbf{Abstract}
%Abstract below

The hypothesis/question and its context, scope and significance are briefly and precisely defined. The main conclusions are brief, precise and clear, noting methods used and key results. The language is clear, precise and easy to understand with no irrelevant information.}
    \renewcommand{\myTitle}{A Survey of Opportunities and Risks of AI Systems in Education}
    \renewcommand{\MyAuthor}{Leon Knorr}
    \renewcommand{\MyDepartment}{Mannheim Master of Datascience}
    \renewcommand{\ID}{1902854}
    \renewcommand{\Keywords}{Education, AI}
    \maketitle
%\vspace{-1.9em}\noindent\rule{\textwidth}{1pt} %add this line if not using abstract
    %\onehalfspacing
%TC:endignore


    \section{Introduction}
    Artificial Intelligence (AI) systems have made significant inroads into the education landscape, promising enhanced learning experiences, improved efficiency, and personalized educational pathways. As the deployment and research of AI technologies in education continues to gather momentum, it is essential to acknowledge that the impact of these systems extends far beyond their immediate advantages and drawbacks \cite{chen_ai_2023}. Engaging in a meaningful and informed discourse about AI in education is imperative to ensure the fair and responsible implementation and development of AI systems in an education setting. In addition, a survey of the Pew Research Center revealed, that a growing share of people are concerned about the development and deployment of AI systems. This includes the fear of potential job loss, privacy concerns and the loss of human connection \cite{nadeem_1_2022}. In this work, Opportunities and Risks are shown based on current state-of-the-art AI systems while addressing, widespread concerns, as well as showing potential problems which should be addressed in the current rapid development of such systems.
    
    \section{Approach}
    In this work, the Opportunities and Risks of AI systems in Education will be discussed in two focus topics:
    \begin{enumerate}
        \item Teaching
        \item Assistive Technologies
    \end{enumerate}
    First, the current state of adoption of AI systems in these categories are laid out. Then, the opportunities of such systems are presented based on real world examples and their future potential. Then, the risks of AI systems and current research in the context of education is discussed, and possible solutions are presented.
    
    \section{The current state of AI and Technology in Education}
    In this section, the current state of AI and Technology is presented. First, the current technologies used in Teaching are laid out. Then, the assistive technologies deployed at schools are presented.

    \subsection{Current Technologies in Teaching}
    Quantifying the actual usage of technological advances in teaching is hard, as it is very subjective to the teacher, as well as the teaching environment. Since 2013, the International Association for the Evaluation of Educational Achievement (IEA) conducts the International Computer and Information Literacy Study in a cycle of 5 years. This study has been designed to capture the current state of how well students and teachers a like are prepared for study, work and life in a increasingly digital world \cite{noauthor_icils_nodate}. Its latest evaluation cycle has been conducted in 2018 with the next one starting in 2023. The study, as well as a German counterpart, showed, that the availability of Technology and digital devices vary greatly between schools, students and teachers \cite{musmann_digitalisierung_2021}. In 2018, most teachers and students didn't have their own portable device for school usage. These circumstances got better over the course of the pandemic. However, still around 60\% of teachers claim to not have access to their own portable digital device for teaching \cite{musmann_digitalisierung_2021}, with 30\% of schools not providing Wireless internet access to teachers or students. This goes inline with the usage numbers for Information and Communication Technology (ICT) during lessons, only 43\% of teachers use word processing or presentation software programs, with 32\% using digital content to supplement school materials \cite{noauthor_icils_nodate}. Besides availability, most teachers claim, that missing confidence in the usage of digital devices and tools is one of the main reasons for the missing integration of ICTs. This applies especially for the usage of learning management systems or Education cloud software such as Moodle or ILIAS \cite{noauthor_icils_nodate}. As a result, the usage of such systems remains very sparse for in presence lectures. An exception is given, for gamification software, e.g. Quizlet or Kahoot, as 60\% of schools report frequent use and good availability \cite{noauthor_icils_nodate}. Because of these circumstances, it comes at no surprise, that the use of photocopiers, paper and overhead projectors are still the most used technologies in teaching \cite{noauthor_202122_nodate}.

    \subsection{Current Assistive Technologies}

    \subsection{Defining AI in Education}
    Defining AI itself is complicated. As of today, there is no precise and universally accepted definition of AI available \cite{Stanford_AI}. One way to define AI is as \enquote{The activity devoted to making machines intelligent, and intelligence is that quality that enables an entitiy to function appropriately and with foresight in its environment.} \cite{Stanford_AI}. This definition depends on whether a system can be credited to function appropriately and with foresight, which can be very subjective. This can range from accepting simple \enquote{dumb} programs such as calculators as AI. In recent times, the understanding of AI has shifted, as machine and deep learning techniques become increasingly available to the public. Today, when talking about an AI, most people think of Computer Vision systems, Large Language Models or intelligent speech recognition agents such as Amazons Alexa or take influence from the entertainment sector \cite{nader_public_2022}.

    The definition of AI in education builds upon this general definition, by defining AI systems in Education (AIED) as systems, that include \enquote{intelligent education, innovative virtual learning, data analysis and prediction} \cite{chen_artificial_2020}. Intelligent education systems are designed to improve learning value and efficiency by providing timely and personalized instruction and feedback to both tutors and students. They include techniques for learning analysis, recommendation and knowledge understanding and acquirement, which uses sets out to use current state-of-the-art data mining, analytics and machine and deep learning technologies \cite{chen_artificial_2020}.

    \section{Opportunities}

    \subsection{Teaching}

    \subsubsection{Collaborative Learning}

    \subsubsection{Student Forum Monitoring}

    \subsubsection{Continuous Assessment}

    \subsubsection{AI Learning Companions}

    \subsubsection{Teaching Assistants}

    \subsection{Assistive Technologies}

    \section{Risks and Obstacles}

    \subsection{Teaching}

    \subsection{Assistive Technologies}

    \section{Conclusion}

%TC:ignore
%\clearpage %add new page for references
    \singlespacing
    \emergencystretch 3em
    \hfuzz 1px
    \printbibliography[heading=bibnumbered]

% \clearpage
% \begin{appendices}

% \section{Here go any appendices!}

% \end{appendices}

%TC:endignore
\end{document}