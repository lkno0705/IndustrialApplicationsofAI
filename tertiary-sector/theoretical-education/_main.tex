\documentclass{Academic}
\usepackage{csquotes}

\addbibresource{references.bib}

\begin{document}
%Easy customisation of title page
%TC:ignore
    \myabstract{\small
\noindent\textbf{Abstract}
%Abstract below

The hypothesis/question and its context, scope and significance are briefly and precisely defined. The main conclusions are brief, precise and clear, noting methods used and key results. The language is clear, precise and easy to understand with no irrelevant information.}
    \renewcommand{\myTitle}{A Survey of Opportunities and Risks of AI Systems in Education}
    \renewcommand{\MyAuthor}{Leon Knorr}
    \renewcommand{\MyDepartment}{Mannheim Master of Datascience}
    \renewcommand{\ID}{1902854}
    \renewcommand{\Keywords}{Education, AI}
    \maketitle
%\vspace{-1.9em}\noindent\rule{\textwidth}{1pt} %add this line if not using abstract
    %\onehalfspacing
%TC:endignore


    \section{Introduction}
    Artificial Intelligence (AI) systems have made significant inroads into the education landscape, promising enhanced learning experiences, improved efficiency, and personalized educational pathways. As the deployment and research of AI technologies in education continues to gather momentum, it is essential to acknowledge that the impact of these systems extends far beyond their immediate advantages and drawbacks \cite{chen_ai_2023}. Engaging in a meaningful and informed discourse about AI in education is imperative to ensure the fair and responsible implementation and development of AI systems in an education setting. In addition, a survey of the Pew Research Center revealed, that a growing share of people are concerned about the development and deployment of AI systems. This includes the fear of potential job loss, privacy concerns and the loss of human connection \cite{nadeem_1_2022}. In this work, Opportunities and Risks are shown based on current state-of-the-art AI systems while addressing, widespread concerns, as well as showing potential problems which should be addressed in the current rapid development of such systems.
    
    \section{Approach}
    In this work, the Opportunities and Risks of AI systems in Education will be discussed in two focus topics:
    \begin{enumerate}
        \item Teaching
        \item Assistive Technologies
    \end{enumerate}
    First, the current state of adoption of AI systems in these categories are laid out. Then, the opportunities of such systems are presented based on real world examples and their future potential. Then, the risks of AI systems and current research in the context of education is discussed, and possible solutions are presented.
    
    \section{The current state of AI and Technology in Education}
    In this section, the current state of AI and Technology is presented. First, the current technologies used in Teaching are laid out. Then, the assistive technologies deployed at schools are presented.

    \subsection{Current Technologies in Teaching}

    \subsection{Current Assistive Technologies}

    \section{Opportunities}

    \subsection{Teaching}

    \subsection{Assistive Technologies}

    \section{Risks}

    \subsection{Teaching}

    \subsection{Assistive Technologies}

    \section{Conclusion}

%TC:ignore
%\clearpage %add new page for references
    \singlespacing
    \emergencystretch 3em
    \hfuzz 1px
    \printbibliography[heading=bibnumbered]

% \clearpage
% \begin{appendices}

% \section{Here go any appendices!}

% \end{appendices}

%TC:endignore
\end{document}