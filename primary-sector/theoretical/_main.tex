\documentclass{Academic}
\begin{document}
%Easy customisation of title page
%TC:ignore
\myabstract{\small
\noindent\textbf{Abstract}
%Abstract below

The hypothesis/question and its context, scope and significance are briefly and precisely defined. The main conclusions are brief, precise and clear, noting methods used and key results. The language is clear, precise and easy to understand with no irrelevant information.}
\renewcommand{\myTitle}{A Survey of State of the Art Crop Rotation Prediction Approaches}
\renewcommand{\MyAuthor}{Leon Knorr}
\renewcommand{\MyDepartment}{Mannheim Master of Datascience}
\renewcommand{\ID}{1902854}
\renewcommand{\Keywords}{Agriculture, AI, Crop Rotation}
\maketitle
%\vspace{-1.9em}\noindent\rule{\textwidth}{1pt} %add this line if not using abstract
\onehalfspacing
%TC:endignore

\section{Introduction}

The hypothesis/question and its context, scope and significance are introduced and explained in a very clear and well-structured way\supercite{einstein}. 
There is an exceptionally well-structured critical review of key literature of direct relevance to the question/hypothesis. Uncertainties and limitations in previous work are carefully discussed such that the gap in knowledge that will be addressed by the project is precisely defined. 
Key concepts are clearly introduced with careful referencing. 
The project aims and objectives are precisely defined so that the significance of the work is clearly evident.

\section{Methods}

The methods used are described clearly and precisely in sufficient detail that an experienced researcher could repeat the work and obtain precisely the same findings. There is no irrelevant background information that could be referenced.
For experimental projects, all materials, processing conditions, sample preparation methods and measurement/characterisation procedures are described precisely with full traceability and referencing of any standard methods.
 For modelling projects, there is precise description of computational methods, procedures for calculating data/predictions including set-up parameters. A careful and precise distinction is made between existing modelling tools/commercial software (appropriately referenced), and new methods created for this project. 


\section{Results}

Graphs are presented with clear and appropriate axes, legends, labels and lines such that key evidence is precise and very convincing.  Bar charts are clear, precise and easy to understand.
Error bars are used consistently and appropriately with precise justification of how they were calculated.
Micrographs are of excellent quality with appropriate contrast and magnification, and clear scale bars.  Multiple magnifications and arrows highlighting key features are used to excellent effect.
Figure and table captions indicate all of the key conditions so that the methods used to obtain the data can be precisely traced to the Methods section, such that the data could be replicated with ease.
Each figure and table is precisely described to highlight the key features that provide evidence for the conclusions.
For modelling projects, clear and precise evidence is provided to show that numerical results are insensitive to input parameters through convergence tests.


\section{Discussion}

\begin{figure}[!htb]%recommended float settings
    \centering
    \includegraphics[width=\textwidth]{example-image-a}% here goes the figure name
    \caption{This is the figure caption.}
    \label{fig:name_me_please}
\end{figure}

Figure \ref{fig:name_me_please}, is an example figure!
The discussion is carefully structured so that precise and robust evidence is provided to underpin each conclusion. This incorporates evidence from the results section (precisely referenced, highlighting key features) and evidence from previous work that was described in the introduction.  
A precise and well-structured critical comparison is made between the experimental/modelling results and previous work with a thorough exploration of all possible sources of uncertainty.
The significance of the findings is precisely described with reference to the question/hypothesis that has been addressed.


\section{Conclusion}

There is a brief and precise description of the context of the work such that it is easy to understand the significance of the conclusions.  
Each conclusion is described precisely, and correlates exactly with the evidence discussed in the discussion section.
The significance of the work is precisely described.
There is no new information that has not been discussed in the rest of the report.
The conclusions section correlates precisely with the abstract, and is easy to understand if taken out of context.


%TC:ignore
%\clearpage %add new page for references
\singlespacing
\emergencystretch 3em
\hfuzz 1px
\printbibliography[heading=bibnumbered]

% \clearpage
% \begin{appendices}

% \section{Here go any appendices!}

% \end{appendices}

%TC:endignore
\end{document}