\documentclass{Academic}
\usepackage{csquotes}

\addbibresource{references.bib}

\begin{document}
%Easy customisation of title page
%TC:ignore
    \myabstract{%\small
%\noindent\textbf{Abstract}
%%Abstract below
%
%The hypothesis/question and its context, scope and significance are briefly and precisely defined. The main conclusions are brief, precise and clear, noting methods used and key results. The language is clear, precise and easy to understand with no irrelevant information.}
    \renewcommand{\myTitle}{WIP: Machine Learning Applications in Silicon / Wafer Manufacturing}
    \renewcommand{\MyAuthor}{Leon Knorr}
    \renewcommand{\MyDepartment}{Mannheim Master of Datascience}
    \renewcommand{\ID}{1902854}
    \renewcommand{\Keywords}{Manufacturing, AI, Silicon, Waver}
    \maketitle
%\vspace{-1.9em}\noindent\rule{\textwidth}{1pt} %add this line if not using abstract
    %\onehalfspacing
%TC:endignore


    \section{Introduction}
    The semiconductor industry plays a pivotal role in driving technological advancements across various sectors, including electronics, communications, and computing. The production of high-quality semiconductor wafers is a critical step, as these wafers serve as the foundation for manufacturing integrated circuits and microchips \cite{batool_systematic_2021}. As the demand for smaller, faster, and more powerful devices continues to grow, the need for efficient and accurate wafer manufacturing processes becomes increasingly important. In recent years, Artificial Intelligence (AI) has emerged as a transformative technology, offering unprecedented opportunities for enhancing the semiconductor manufacturing workflow \cite{batool_systematic_2021,susto_automatic_2012}.
    Especially machine learning (ML) algorithms, have the potential to revolutionize semiconductor wafer manufacturing by providing intelligent solutions for quality control and process optimization. For example, traditionally, wafer inspection and defect detection have been labor-intensive and time-consuming tasks performed by human operators, leading to potential errors and inconsistencies. In addition, human operators operate without using every bit of the vast amounts of data available in a wafer manufacturing fab. Each machine of the manufacturing process is equipped with hundreds of sensors, that measure the conditions of the current process \cite{kim_machine_2012}. However, the integration of AI into the manufacturing process can significantly improve these aspects, enabling real-time, automated, and highly accurate detection and classification of defects \cite{yuan-fu_deep_2019-1}. Aside from defect detection, semiconductor manufacturing poses a lot of potential applications where AI can thrive and enhance the capabilities of a manufacturing plant. This paper aims to explore various applications of AI and machine learning in a semiconductor manufacturing process by explaining them and discussing their impact on the industry.

    \section{Approach}
    First, the manufacturing process of semiconductors is defined and the performance indicators for semiconductor manufacturing factories are laid out. Afterwards, a survey of current applications of AI in the manufacturing process is presented. During the survey, each field of application is explained, and different approaches are laid out and discussed.

    \section{Semiconductor Manufacturing}
    In this section, an overview of the process of semiconductor manufacturing, as well as important definitions are presented.

    \subsection{The Manufacturing Process}
    The manufacturing of semiconductors is a highly delicate and complicated procedure \cite{noauthor_machine_nodate}. It consists of hundreds of processes, where the input materials are processed according to the given recipes. These processes can be summarized into four stages:
    \begin{itemize}
        \item Wafer Formation
        \item Front End Processing
        \item Testing
        \item Packaging
    \end{itemize}
    Each of these stages is presented in detail in the following section. \\
    It is important to note, that this section only presents the basics of the manufacturing process and that it could've been altered or optimized for different recipes or to improve yields. As a result, some steps mentioned later on during the discussion of AI applications in this industry may not be listed here but are explained as needed.

    \subsubsection{Wafer Formation}
    During Wafer Formation, a so-called wafer is created. A wafer is a thin slice, often of a round shape, of extremely pure crystalline material. Usually silicon crystals are used. These wafers serve as the foundational material for microelectronic devices \cite{susto_automatic_2012}. Each Wafer will hold multiple dies, where a die refers to one specific microchip. After a wafer is formed, it is organized into \enquote{Lots}. Each lot contains up to 25 wafers, which are processed together, using the same recipe and configuration at each step in the applied recipe \cite{susto_automatic_2012}. After Wafer Formation, the formed wafers will continue to Front End Processing.

    \subsubsection{Front End Processing}
    Front End Processing, encompasses the creation of transistors on a silicon wafer. It is performed in special \enquote{clean rooms}, where dust, vapor and other pollutants are kept at a minimum. This is achieved through air filtering and strict access policies \cite{susto_automatic_2012}. Each step in the Front End Process is repeated multiple times, to produce multiple interconnected layers on a wafers surface. \\
    First, a Lot of wafers runs through \textit{Wafer-cleaning}. In this step, each wafer is cleaned, such that is has a surface which is as smooth as possible. This is important, as following processes have strict requirements concerning surface smoothness and particle contamination. If one of these requirements are not fulfilled as good as possible, it can lead to defective dies later on in the process and reduce yields \cite{susto_automatic_2012}.\\
    After Wafer-cleaning, each wafer is run through a process called \textit{deposition}. During deposition, a dielectric and polysilicon film is applied to each individual wafer \cite{susto_automatic_2012}. This film serves as the foundation for \textit{Photo Lithography}. During Photo Lithography, an ultraviolet radiation is transmitted through a \textit{photomask} to create circuit patterns according to the recipes. The radiation travels through the non-opaque parts of the mask and reacts with the previously applied film, which is coated onto the wafers surface as a result \cite{susto_automatic_2012}. As this mask essentially burns in the required circuits into the film, the mask has to be aligned as accurately as possible, with only very small tolerances.\\
    After Photo Lithography, the lot of wafers are passed on to \textit{Etching}. During Etching, unused chemicals and layers are removed from a wafer. In the etching process, the circuits patterned onto the wafer using the resistant film, during Photo Lithography, serve as a masking material which resits the etchant. As a result, the patterned circuits are transferred onto the wafer by removing unnecessary material, layers and chemicals \cite{susto_automatic_2012}.\\
    After Front End Processing is complete, and all needed circuits are transferred onto the actual wafers, each wafer lot proceeds to Testing.

    \subsubsection{Testing}
    During Testing, each circuit is tested for functionality and defects. Two different types of tests are carried out:
    \begin{description}
        \item[Parametric Tests] Parametric Tests are performed to monitor the efficiency of the process and the quality of the design. It is measured on ad-hoc structured prepared onto the device, and consists of electrical measurements of physical quantities, such as impedance, capacitance and resistance \cite{susto_automatic_2012}.
        \item[Electrical Tests] Electrical Tests, verify that each die works within specifications, and that its behavior is consistent. If a die performs out of spec, it is marked with a small dot of ink and the passing / non-passing information is stored in a \textit{wafermap} \cite{susto_automatic_2012}.
    \end{description}
    After testing, the manufactured wafers proceed to \textit{Packaging}.

    \subsubsection{Packaging}
    In Packaging, the wafers are sawed into pieces, which only include one die at a time. Then, electrical connections, such as Pins etc., and Integrated Heat Spreaders (IHS) are added to the chip to protect it from mechanical and environment stress, as well providing a proper thermal path for the heat the chip generates under load \cite{susto_automatic_2012}.

    \subsection{Performance Indicators of Semiconductor Fabs}
    As with any manufacturing process, semiconductor factories have certain indicators, that represent their efficiency, competitiveness and success. The two main Indicators are the Yield and costs of the manufacturing process \cite{yuan-fu_deep_2019-1,batool_systematic_2021,noauthor_machine_nodate}. The yield of the process refers to the total amount of chips produced for a given recipe for a given amount of raw material. Yields of a certain ordered chip can vary a lot depending on the state of the equipment, the raw materials, procedures applied etc. Improving Yield, will also decrease costs, as the process gets more efficient and less material has to be used to produce the same number of chips, or if seen the other way around, more chips can be produced for the same amount of material and costs. Improving yield, will also increase the competitiveness of a given factory, as offered pricing can be adjusted accordingly \cite{batool_systematic_2021}. As yield is influenced by a lot of measurable factors, which in return involves a lot of data, applications around yield optimization are the primary ares of use for AI applications \cite{jiang_novel_2020}. There are two types of yield:
    \begin{description}
        \item[Line Yield] Which refers to the fraction of wafers that reach the final electrical test \cite{noauthor_machine_nodate}.
        \item[Die Yield] Die Yield on the other hand is the fraction of dice on yielding wafers, that are not discarded before reaching assembly and final test \cite{noauthor_machine_nodate}. 
    \end{description}
    Besides Yield, an important quantity is called the Cycle Time. Cycle time refers to the total amount of time a wafer needs to be processed from start to finish. The shorter the cycle time, the better the yields and lower the costs \cite{wang_fuzzy_2021}.

    In reality however, a lot more indicators are present to define the competitiveness of a semiconductor factory, such as the smallest available node size, which influences the power efficiency of a chip and allows the packaging of a lot more transistors onto the same die size.

    \section{AI Applications}
    In this section current applications of AI in semiconductor manufacturing are outlined and discussed.\\
    As mentioned earlier, most AI applications that are specific to the domain of semiconductor manufacturing are applied during front end processing. The goal during this stage is to optimize yields by detecting defects as early as possible, minimizing cycle times, optimizing maintenance of the used equipment and predicting the yield of a configuration early on \cite{batool_systematic_2021,wang_fuzzy_2021,jiang_novel_2020,irani_applying_1993,kim_machine_2012}. In summary then, all applications influence the yield management strategy of a semiconductor factory in a certain way, in order to make it as efficient as possible.

    \subsection{Defect Detection}
    The first application of AI in semiconductor manufacturing is defect detection. As the manufacturing process involves hundreds of sequential and recurring processes, it is essential to detect a fault as early as possible. Even the tiniest scratch or particle contamination of a wafer can lead to lower yields \cite{kim_machine_2012}. Over the years, three main categories for defect detection during the manufacturing process have emerged:
    \begin{itemize}
        \item Virtual Metrology
        \item Machine Learning approaches
        \item Deep Learning Applications
    \end{itemize}

    \subsubsection{Virtual Metrology}
    One way to detect faults is to employ a metrology step after each step in the manufacturing process. During this step, metrology equipment is used to measure physical variables of a given wafer. However, this step is time-consuming and expensive. Thus, it increases Cycle Times and costs \cite{shim_domain-adaptive_2022}. As a result, semiconductor factories use the similarity property of wafers in a lot in order to reduce the amount of time spend on metrology. Instead of measuring each wafer in a lot, a fraction of wafers is sampled from the lot and inspected accordingly \cite{kim_machine_2012}. However, non-inspected wafers may still have quality issues, even though the sampled wafer(s) passed the inspection. To overcome this gap, research introduced the concept of Virtual Metrology (VM). VM estimates metrology values for all wafers in a lot, using a predictive regression model \cite{kim_machine_2012}. These models are build using previous real metrology data, as well as data from the preceding process, to infer the actual metrology values with reduce costs \cite{shim_domain-adaptive_2022,kim_machine_2012}. Common techniques in Virtual metrology are domain adaptive active learning, Neural Networks. Neural Networks are used to model metrology as a non-linear combination between the tool and logistic data and the target, which is the unlabeled wafer. They have been shown to be powerful approximators for VM but are hard to train because of VMs high dimensionality \cite{susto_automatic_2012}. In addition, the results often lack interpretability. \\
    In domain adaptive active learning aims at reducing the problem of class imbalance for VM models. As only a fraction of wafers is actually inspected using metrology equipment, the number of accurately labeled wafers is small. The amount of unlabeled wafers on the other hand is huge. In order to reach the desired prediction accuracy, it is necessary to increase the amount of labeled wafers by annotating additional wafers, which again is time-consuming and expensive \cite{shim_domain-adaptive_2022}. During adaptive learning, the VM model iteratively selects wafers to be annotated and uses them to increase the prediction accuracy. However, as manufacturing processes can change on a daily basis, and each change can alter the distribution of process measurements, a new model has to be built every time new equipment is introduced, or a significant change is applied to the manufacturing process \cite{shim_domain-adaptive_2022}. In this situation no labeled wafers are readily available, and thus, the adaptive learning process can't be started. To overcome this problem, Shim and Kang \cite{shim_domain-adaptive_2022} proposed domain-adaptive active learning. This approaches uses data from existing equipment, as the source domain. Then applied, unsupervised domain adaption to unlabeled wafers as the target and labeled wafers as the source domain to extract domain invariant features. This process allows the construction of a base VM model, which can be used to initialize active learning \cite{shim_domain-adaptive_2022}. Active learning then proceeds to iteratively update the model and tune it for the current domain (new process and new equipment).
    From a business perspective VM models have to have two properties. They have to have low compute times, as they always need to be updated and their primary task is to reduce cycle times. And, the results have to be interpretable. This is important, as it allows the identification of the most important variables, which can be used to enhance the manufacturing process and reduce defects. According to Susto et al. especially the latter is challenging for current research \cite{susto_automatic_2012}.
    

    \subsubsection{Machine Learning}

    \subsubsection{Deep Learning}

    \subsection{Yield Prediction}

    \subsection{Cycle Time Optimization}

    \subsection{Predictive Maintenance}

    \section{Conclusion}

%TC:ignore
%\clearpage %add new page for references
    \singlespacing
    \emergencystretch 3em
    \hfuzz 1px
    \printbibliography[heading=bibnumbered]

% \clearpage
% \begin{appendices}

% \section{Here go any appendices!}

% \end{appendices}

%TC:endignore
\end{document}